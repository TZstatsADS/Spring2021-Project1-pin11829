% Options for packages loaded elsewhere
\PassOptionsToPackage{unicode}{hyperref}
\PassOptionsToPackage{hyphens}{url}
%
\documentclass[
]{article}
\usepackage{lmodern}
\usepackage{amsmath}
\usepackage{ifxetex,ifluatex}
\ifnum 0\ifxetex 1\fi\ifluatex 1\fi=0 % if pdftex
  \usepackage[T1]{fontenc}
  \usepackage[utf8]{inputenc}
  \usepackage{textcomp} % provide euro and other symbols
  \usepackage{amssymb}
\else % if luatex or xetex
  \usepackage{unicode-math}
  \defaultfontfeatures{Scale=MatchLowercase}
  \defaultfontfeatures[\rmfamily]{Ligatures=TeX,Scale=1}
\fi
% Use upquote if available, for straight quotes in verbatim environments
\IfFileExists{upquote.sty}{\usepackage{upquote}}{}
\IfFileExists{microtype.sty}{% use microtype if available
  \usepackage[]{microtype}
  \UseMicrotypeSet[protrusion]{basicmath} % disable protrusion for tt fonts
}{}
\makeatletter
\@ifundefined{KOMAClassName}{% if non-KOMA class
  \IfFileExists{parskip.sty}{%
    \usepackage{parskip}
  }{% else
    \setlength{\parindent}{0pt}
    \setlength{\parskip}{6pt plus 2pt minus 1pt}}
}{% if KOMA class
  \KOMAoptions{parskip=half}}
\makeatother
\usepackage{xcolor}
\IfFileExists{xurl.sty}{\usepackage{xurl}}{} % add URL line breaks if available
\IfFileExists{bookmark.sty}{\usepackage{bookmark}}{\usepackage{hyperref}}
\hypersetup{
  pdftitle={Money War in Election},
  hidelinks,
  pdfcreator={LaTeX via pandoc}}
\urlstyle{same} % disable monospaced font for URLs
\usepackage[margin=1in]{geometry}
\usepackage{color}
\usepackage{fancyvrb}
\newcommand{\VerbBar}{|}
\newcommand{\VERB}{\Verb[commandchars=\\\{\}]}
\DefineVerbatimEnvironment{Highlighting}{Verbatim}{commandchars=\\\{\}}
% Add ',fontsize=\small' for more characters per line
\usepackage{framed}
\definecolor{shadecolor}{RGB}{248,248,248}
\newenvironment{Shaded}{\begin{snugshade}}{\end{snugshade}}
\newcommand{\AlertTok}[1]{\textcolor[rgb]{0.94,0.16,0.16}{#1}}
\newcommand{\AnnotationTok}[1]{\textcolor[rgb]{0.56,0.35,0.01}{\textbf{\textit{#1}}}}
\newcommand{\AttributeTok}[1]{\textcolor[rgb]{0.77,0.63,0.00}{#1}}
\newcommand{\BaseNTok}[1]{\textcolor[rgb]{0.00,0.00,0.81}{#1}}
\newcommand{\BuiltInTok}[1]{#1}
\newcommand{\CharTok}[1]{\textcolor[rgb]{0.31,0.60,0.02}{#1}}
\newcommand{\CommentTok}[1]{\textcolor[rgb]{0.56,0.35,0.01}{\textit{#1}}}
\newcommand{\CommentVarTok}[1]{\textcolor[rgb]{0.56,0.35,0.01}{\textbf{\textit{#1}}}}
\newcommand{\ConstantTok}[1]{\textcolor[rgb]{0.00,0.00,0.00}{#1}}
\newcommand{\ControlFlowTok}[1]{\textcolor[rgb]{0.13,0.29,0.53}{\textbf{#1}}}
\newcommand{\DataTypeTok}[1]{\textcolor[rgb]{0.13,0.29,0.53}{#1}}
\newcommand{\DecValTok}[1]{\textcolor[rgb]{0.00,0.00,0.81}{#1}}
\newcommand{\DocumentationTok}[1]{\textcolor[rgb]{0.56,0.35,0.01}{\textbf{\textit{#1}}}}
\newcommand{\ErrorTok}[1]{\textcolor[rgb]{0.64,0.00,0.00}{\textbf{#1}}}
\newcommand{\ExtensionTok}[1]{#1}
\newcommand{\FloatTok}[1]{\textcolor[rgb]{0.00,0.00,0.81}{#1}}
\newcommand{\FunctionTok}[1]{\textcolor[rgb]{0.00,0.00,0.00}{#1}}
\newcommand{\ImportTok}[1]{#1}
\newcommand{\InformationTok}[1]{\textcolor[rgb]{0.56,0.35,0.01}{\textbf{\textit{#1}}}}
\newcommand{\KeywordTok}[1]{\textcolor[rgb]{0.13,0.29,0.53}{\textbf{#1}}}
\newcommand{\NormalTok}[1]{#1}
\newcommand{\OperatorTok}[1]{\textcolor[rgb]{0.81,0.36,0.00}{\textbf{#1}}}
\newcommand{\OtherTok}[1]{\textcolor[rgb]{0.56,0.35,0.01}{#1}}
\newcommand{\PreprocessorTok}[1]{\textcolor[rgb]{0.56,0.35,0.01}{\textit{#1}}}
\newcommand{\RegionMarkerTok}[1]{#1}
\newcommand{\SpecialCharTok}[1]{\textcolor[rgb]{0.00,0.00,0.00}{#1}}
\newcommand{\SpecialStringTok}[1]{\textcolor[rgb]{0.31,0.60,0.02}{#1}}
\newcommand{\StringTok}[1]{\textcolor[rgb]{0.31,0.60,0.02}{#1}}
\newcommand{\VariableTok}[1]{\textcolor[rgb]{0.00,0.00,0.00}{#1}}
\newcommand{\VerbatimStringTok}[1]{\textcolor[rgb]{0.31,0.60,0.02}{#1}}
\newcommand{\WarningTok}[1]{\textcolor[rgb]{0.56,0.35,0.01}{\textbf{\textit{#1}}}}
\usepackage{graphicx}
\makeatletter
\def\maxwidth{\ifdim\Gin@nat@width>\linewidth\linewidth\else\Gin@nat@width\fi}
\def\maxheight{\ifdim\Gin@nat@height>\textheight\textheight\else\Gin@nat@height\fi}
\makeatother
% Scale images if necessary, so that they will not overflow the page
% margins by default, and it is still possible to overwrite the defaults
% using explicit options in \includegraphics[width, height, ...]{}
\setkeys{Gin}{width=\maxwidth,height=\maxheight,keepaspectratio}
% Set default figure placement to htbp
\makeatletter
\def\fps@figure{htbp}
\makeatother
\setlength{\emergencystretch}{3em} % prevent overfull lines
\providecommand{\tightlist}{%
  \setlength{\itemsep}{0pt}\setlength{\parskip}{0pt}}
\setcounter{secnumdepth}{-\maxdimen} % remove section numbering
\ifluatex
  \usepackage{selnolig}  % disable illegal ligatures
\fi

\title{Money War in Election}
\author{}
\date{\vspace{-2.5em}}

\begin{document}
\maketitle

\hypertarget{introduction}{%
\section{1. Introduction}\label{introduction}}

\hypertarget{proposal}{%
\subsubsection{Proposal}\label{proposal}}

Money occupies a significant part of people's lives, and it is also
closely related to national policies. If you can make more money for the
people, you have a better chance of earning everyone's votes. Therefore,
we will use the ANES dataset to delve into this topic by solving the
following three questions: Are people care about their money and
national economics? How people's financial condition affect their vote?
What economic policy should the candidate use to influence the election?

\hypertarget{dataset}{%
\subsubsection{Dataset}\label{dataset}}

\href{https://electionstudies.org/data-center/2020-exploratory-testing-survey/}{American
National Election Studies (ANES) - 2020 Exploratory Testing Survey}

\hypertarget{enviroment}{%
\subsubsection{Enviroment}\label{enviroment}}

This report is prepared with the following environmental settings.

\begin{Shaded}
\begin{Highlighting}[]
\FunctionTok{print}\NormalTok{(R.version)}
\end{Highlighting}
\end{Shaded}

\begin{verbatim}
##                _                           
## platform       x86_64-apple-darwin17.0     
## arch           x86_64                      
## os             darwin17.0                  
## system         x86_64, darwin17.0          
## status                                     
## major          4                           
## minor          0.3                         
## year           2020                        
## month          10                          
## day            10                          
## svn rev        79318                       
## language       R                           
## version.string R version 4.0.3 (2020-10-10)
## nickname       Bunny-Wunnies Freak Out
\end{verbatim}

\hypertarget{questions}{%
\section{2. Questions}\label{questions}}

\hypertarget{we-will-use-three-questions-step-by-step-to-bring-out-our-story-the-relation-between-money-and-election.}{%
\paragraph{We will use three questions step by step to bring out our
story -- the relation between money and
election.}\label{we-will-use-three-questions-step-by-step-to-bring-out-our-story-the-relation-between-money-and-election.}}

\begin{center}\rule{0.5\linewidth}{0.5pt}\end{center}

\hypertarget{question1-the-annual-income-distribution-of-the-american-family}{%
\subsection{Question1: The annual income distribution of the American
family?}\label{question1-the-annual-income-distribution-of-the-american-family}}

First, we investigate people's views on the current economic
environment. How do they feel? And what is their financial situation?

\begin{verbatim}
## # A tibble: 1,067 x 470
##       V1 StartDate EndDate `_v1` RecordedDate ResponseId qmetadata_Brows~
##    <dbl> <chr>     <chr>   <dbl> <chr>        <chr>      <chr>           
##  1  2875 4/16/202~ 4/16/2~  3170 4/16/2020 1~ R_1jdLlxo~ Firefox         
##  2   859 4/17/202~ 4/17/2~  3490 4/17/2020 2~ R_ClTXw6W~ Chrome          
##  3  2031 4/16/202~ 4/16/2~  3885 4/16/2020 1~ R_3sjr4ze~ Chrome          
##  4   587 4/16/202~ 4/16/2~  2330 4/16/2020 1~ R_21H8jh0~ Chrome          
##  5   183 4/14/202~ 4/14/2~  1435 4/14/2020 1~ R_25A8BQN~ Chrome          
##  6  1174 4/10/202~ 4/10/2~  3228 4/10/2020 2~ R_3HZAmYE~ Safari          
##  7  3006 4/14/202~ 4/14/2~  5311 4/14/2020 1~ R_3J7PmwG~ Safari iPad     
##  8  2678 4/14/202~ 4/14/2~  2302 4/14/2020 1~ R_3gMf5Mh~ Chrome          
##  9  1303 4/11/202~ 4/11/2~  1394 4/11/2020 1~ R_Uxvn1hC~ Safari iPhone   
## 10  1205 4/11/202~ 4/11/2~  1443 4/11/2020 1~ R_3dQHvKZ~ Chrome          
## # ... with 1,057 more rows, and 463 more variables: qmetadata_Version <chr>,
## #   `_v2` <chr>, qmetadata_Resolution <chr>, follow <dbl>, reg1 <dbl>,
## #   votemail1a <dbl>, votemail1b <dbl>, votecount <dbl>, votemail2 <dbl>,
## #   voterid1 <dbl>, voterid2 <dbl>, turnout16a <dbl>, turnout16a1 <dbl>,
## #   turnout16b <dbl>, vote16 <dbl>, hopeful <dbl>, afraid <dbl>,
## #   outraged <dbl>, angry <dbl>, happy <dbl>, worried <dbl>, proud <dbl>,
## #   irritated <dbl>, nervous <dbl>, meeting <dbl>, moneyorg <dbl>,
## #   protest <dbl>, online <dbl>, persuade <dbl>, button <dbl>, moneycand <dbl>,
## #   argument <dbl>, particip_none <dbl>, talk1 <dbl>, talk2 <dbl>, talk3 <dbl>,
## #   fttrump1 <dbl>, ftobama1 <dbl>, ftbiden1 <dbl>, ftwarren1 <dbl>,
## #   ftsanders1 <dbl>, ftbuttigieg1 <dbl>, ftharris1 <dbl>, ftklobuchar1 <dbl>,
## #   ftpence1 <dbl>, ftyang1 <dbl>, ftpelosi1 <dbl>, ftrubio1 <dbl>,
## #   ftocasioc1 <dbl>, fthaley1 <dbl>, ftthomas1 <dbl>, ftfauci1 <dbl>,
## #   ftblack <dbl>, ftwhite <dbl>, fthisp <dbl>, ftasian <dbl>, ftillegal <dbl>,
## #   ftfeminists <dbl>, ftmetoo <dbl>, fttransppl <dbl>, ftsocialists <dbl>,
## #   ftcapitalists <dbl>, ftbigbusiness <dbl>, ftlaborunions <dbl>,
## #   ftrepublicanparty <dbl>, ftdemocraticparty <dbl>, primaryvote <dbl>,
## #   vote20jb <dbl>, vote20bs <dbl>, cvote2020 <dbl>, apppres7 <dbl>,
## #   frnpres7 <dbl>, immpres7 <dbl>, econpres7 <dbl>, covidpres7 <dbl>,
## #   healthcarepres7 <dbl>, check <dbl>, dtleader1 <dbl>, dtleader2 <dbl>,
## #   dtcares <dbl>, dtdignif <dbl>, dthonest <dbl>, dtauth <dbl>, dtdiv <dbl>,
## #   dtknow <dbl>, jbleader1 <dbl>, jbleader2 <dbl>, jbcares <dbl>,
## #   jbdignif <dbl>, jbhonest <dbl>, jbauth <dbl>, jbdiv <dbl>, jbknow <dbl>,
## #   bsleader1 <dbl>, bsleader2 <dbl>, bscares <dbl>, bsdignif <dbl>,
## #   bshonest <dbl>, bsauth <dbl>, bsdiv <dbl>, ...
\end{verbatim}

\begin{verbatim}
## # A tibble: 1,003 x 470
##       V1 StartDate EndDate `_v1` RecordedDate ResponseId qmetadata_Brows~
##    <dbl> <chr>     <chr>   <dbl> <chr>        <chr>      <chr>           
##  1  1451 4/12/202~ 4/12/2~  1131 4/12/2020 1~ R_3oBJppI~ Chrome          
##  2  1611 4/13/202~ 4/13/2~  2563 4/13/2020 1~ R_3fWc87B~ Chrome          
##  3  2813 4/10/202~ 4/10/2~  2820 4/10/2020 2~ R_2aleErU~ Safari iPhone   
##  4  2893 4/14/202~ 4/14/2~  3309 4/14/2020 1~ R_1etVpGg~ Chrome          
##  5  1845 4/15/202~ 4/15/2~  2456 4/15/2020 1~ R_2TMYKvD~ Chrome          
##  6   810 4/18/202~ 4/18/2~  3175 4/18/2020 1~ R_2167E4P~ Chrome          
##  7   581 4/15/202~ 4/15/2~  2323 4/15/2020 1~ R_3R7UhjO~ Safari          
##  8  1443 4/12/202~ 4/12/2~   724 4/12/2020 1~ R_1NDKttV~ Chrome          
##  9   333 4/15/202~ 4/15/2~  1788 4/15/2020 1~ R_Ua2PmiN~ Chrome          
## 10   659 4/13/202~ 4/13/2~  2503 4/13/2020 2~ R_3gMLM4H~ Chrome          
## # ... with 993 more rows, and 463 more variables: qmetadata_Version <chr>,
## #   `_v2` <chr>, qmetadata_Resolution <chr>, follow <dbl>, reg1 <dbl>,
## #   votemail1a <dbl>, votemail1b <dbl>, votecount <dbl>, votemail2 <dbl>,
## #   voterid1 <dbl>, voterid2 <dbl>, turnout16a <dbl>, turnout16a1 <dbl>,
## #   turnout16b <dbl>, vote16 <dbl>, hopeful <dbl>, afraid <dbl>,
## #   outraged <dbl>, angry <dbl>, happy <dbl>, worried <dbl>, proud <dbl>,
## #   irritated <dbl>, nervous <dbl>, meeting <dbl>, moneyorg <dbl>,
## #   protest <dbl>, online <dbl>, persuade <dbl>, button <dbl>, moneycand <dbl>,
## #   argument <dbl>, particip_none <dbl>, talk1 <dbl>, talk2 <dbl>, talk3 <dbl>,
## #   fttrump1 <dbl>, ftobama1 <dbl>, ftbiden1 <dbl>, ftwarren1 <dbl>,
## #   ftsanders1 <dbl>, ftbuttigieg1 <dbl>, ftharris1 <dbl>, ftklobuchar1 <dbl>,
## #   ftpence1 <dbl>, ftyang1 <dbl>, ftpelosi1 <dbl>, ftrubio1 <dbl>,
## #   ftocasioc1 <dbl>, fthaley1 <dbl>, ftthomas1 <dbl>, ftfauci1 <dbl>,
## #   ftblack <dbl>, ftwhite <dbl>, fthisp <dbl>, ftasian <dbl>, ftillegal <dbl>,
## #   ftfeminists <dbl>, ftmetoo <dbl>, fttransppl <dbl>, ftsocialists <dbl>,
## #   ftcapitalists <dbl>, ftbigbusiness <dbl>, ftlaborunions <dbl>,
## #   ftrepublicanparty <dbl>, ftdemocraticparty <dbl>, primaryvote <dbl>,
## #   vote20jb <dbl>, vote20bs <dbl>, cvote2020 <dbl>, apppres7 <dbl>,
## #   frnpres7 <dbl>, immpres7 <dbl>, econpres7 <dbl>, covidpres7 <dbl>,
## #   healthcarepres7 <dbl>, check <dbl>, dtleader1 <dbl>, dtleader2 <dbl>,
## #   dtcares <dbl>, dtdignif <dbl>, dthonest <dbl>, dtauth <dbl>, dtdiv <dbl>,
## #   dtknow <dbl>, jbleader1 <dbl>, jbleader2 <dbl>, jbcares <dbl>,
## #   jbdignif <dbl>, jbhonest <dbl>, jbauth <dbl>, jbdiv <dbl>, jbknow <dbl>,
## #   bsleader1 <dbl>, bsleader2 <dbl>, bscares <dbl>, bsdignif <dbl>,
## #   bshonest <dbl>, bsauth <dbl>, bsdiv <dbl>, ...
\end{verbatim}

\begin{verbatim}
## # A tibble: 1,006 x 470
##       V1 StartDate EndDate `_v1` RecordedDate ResponseId qmetadata_Brows~
##    <dbl> <chr>     <chr>   <dbl> <chr>        <chr>      <chr>           
##  1  1366 4/11/202~ 4/11/2~  4856 4/11/2020 2~ R_1l4Vkmz~ Safari iPhone   
##  2  1194 4/11/202~ 4/11/2~  2971 4/11/2020 9~ R_2xW7qWx~ Chrome          
##  3  2356 4/14/202~ 4/14/2~  1550 4/14/2020 1~ R_3qyz3yb~ Safari          
##  4  2049 4/16/202~ 4/16/2~  4661 4/16/2020 1~ R_3lSMMSM~ Safari iPhone   
##  5  2050 4/16/202~ 4/16/2~  4169 4/16/2020 1~ R_1pyTHnN~ Chrome          
##  6   100 4/13/202~ 4/13/2~  1207 4/13/2020 2~ R_3J30lTY~ Chrome          
##  7  1917 4/15/202~ 4/15/2~  1722 4/15/2020 1~ R_DBtUfO7~ Chrome          
##  8  2007 4/16/202~ 4/16/2~  1624 4/16/2020 1~ R_2bNwgpJ~ Safari iPhone   
##  9  2061 4/16/202~ 4/16/2~  4068 4/16/2020 1~ R_3lVtWyb~ Chrome          
## 10  2777 4/14/202~ 4/14/2~  2642 4/14/2020 1~ R_1K2rgZi~ Chrome          
## # ... with 996 more rows, and 463 more variables: qmetadata_Version <chr>,
## #   `_v2` <chr>, qmetadata_Resolution <chr>, follow <dbl>, reg1 <dbl>,
## #   votemail1a <dbl>, votemail1b <dbl>, votecount <dbl>, votemail2 <dbl>,
## #   voterid1 <dbl>, voterid2 <dbl>, turnout16a <dbl>, turnout16a1 <dbl>,
## #   turnout16b <dbl>, vote16 <dbl>, hopeful <dbl>, afraid <dbl>,
## #   outraged <dbl>, angry <dbl>, happy <dbl>, worried <dbl>, proud <dbl>,
## #   irritated <dbl>, nervous <dbl>, meeting <dbl>, moneyorg <dbl>,
## #   protest <dbl>, online <dbl>, persuade <dbl>, button <dbl>, moneycand <dbl>,
## #   argument <dbl>, particip_none <dbl>, talk1 <dbl>, talk2 <dbl>, talk3 <dbl>,
## #   fttrump1 <dbl>, ftobama1 <dbl>, ftbiden1 <dbl>, ftwarren1 <dbl>,
## #   ftsanders1 <dbl>, ftbuttigieg1 <dbl>, ftharris1 <dbl>, ftklobuchar1 <dbl>,
## #   ftpence1 <dbl>, ftyang1 <dbl>, ftpelosi1 <dbl>, ftrubio1 <dbl>,
## #   ftocasioc1 <dbl>, fthaley1 <dbl>, ftthomas1 <dbl>, ftfauci1 <dbl>,
## #   ftblack <dbl>, ftwhite <dbl>, fthisp <dbl>, ftasian <dbl>, ftillegal <dbl>,
## #   ftfeminists <dbl>, ftmetoo <dbl>, fttransppl <dbl>, ftsocialists <dbl>,
## #   ftcapitalists <dbl>, ftbigbusiness <dbl>, ftlaborunions <dbl>,
## #   ftrepublicanparty <dbl>, ftdemocraticparty <dbl>, primaryvote <dbl>,
## #   vote20jb <dbl>, vote20bs <dbl>, cvote2020 <dbl>, apppres7 <dbl>,
## #   frnpres7 <dbl>, immpres7 <dbl>, econpres7 <dbl>, covidpres7 <dbl>,
## #   healthcarepres7 <dbl>, check <dbl>, dtleader1 <dbl>, dtleader2 <dbl>,
## #   dtcares <dbl>, dtdignif <dbl>, dthonest <dbl>, dtauth <dbl>, dtdiv <dbl>,
## #   dtknow <dbl>, jbleader1 <dbl>, jbleader2 <dbl>, jbcares <dbl>,
## #   jbdignif <dbl>, jbhonest <dbl>, jbauth <dbl>, jbdiv <dbl>, jbknow <dbl>,
## #   bsleader1 <dbl>, bsleader2 <dbl>, bscares <dbl>, bsdignif <dbl>,
## #   bshonest <dbl>, bsauth <dbl>, bsdiv <dbl>, ...
\end{verbatim}

\hypertarget{people-are-very-anxious-about-the-current-economic-environment}{%
\subsubsection{People are very anxious about the current economic
environment}\label{people-are-very-anxious-about-the-current-economic-environment}}

The bar graph below shows that many people are extremely worried about
the economic environment, and their emotions are in a very extreme
state.

\includegraphics{Proj_1_files/figure-latex/worry-1.pdf}

Therefore, the candidate who can \textbf{propose a reasonable policy,
and somehow fix the economic environment}, the candidate will earn
people's attention and win the election!

\begin{center}\rule{0.5\linewidth}{0.5pt}\end{center}

\hypertarget{the-financial-situation-of-the-people-in-the-survey}{%
\subsubsection{The financial situation of the people in the
survey}\label{the-financial-situation-of-the-people-in-the-survey}}

To see a particular political standpoint and how the economic policies
affect people's voting, we first take an in-depth look into people's
annual income.

\includegraphics{Proj_1_files/figure-latex/year barplot-1.pdf} As we
seen above, we can find that the median of the annual income is:

\begin{verbatim}
## [1] 11
\end{verbatim}

\hypertarget{which-is-range-between-50000---54999.-moreover-we-found-that-the-distribution-of-annual-income-presented-in-a-v-shape-means-most-people-are-either-rich-or-poor.-how-to-make-society-more-balanced-is-one-of-the-most-important-issues-faced-by-every-presidential-candidate.}{%
\subsection{\texorpdfstring{Which is range between \$50,000 - \$54,999.
Moreover, we found that the distribution of annual income presented in a
\textbf{V} shape means most people are either rich or poor. How to make
society more balanced is one of the most important issues faced by every
presidential
candidate.}{Which is range between \$50,000 - \$54,999. Moreover, we found that the distribution of annual income presented in a V shape means most people are either rich or poor. How to make society more balanced is one of the most important issues faced by every presidential candidate.}}\label{which-is-range-between-50000---54999.-moreover-we-found-that-the-distribution-of-annual-income-presented-in-a-v-shape-means-most-people-are-either-rich-or-poor.-how-to-make-society-more-balanced-is-one-of-the-most-important-issues-faced-by-every-presidential-candidate.}}

\hypertarget{question2-according-to-different-financial-conditions-which-candidate-and-political-party-they-will-vote}{%
\subsection{Question2: According to different financial conditions,
which candidate and political party they will
vote?}\label{question2-according-to-different-financial-conditions-which-candidate-and-political-party-they-will-vote}}

To make the statement more precise, we will use the first question to
analyze in the next step. In our second question, we want to see
people's voting preference in every economic range. Try to figure out
each presidential candidate needs to focus on which group to win the
election.

\hypertarget{the-ratio-of-president-preference-in-each-economic-range}{%
\subsubsection{The ratio of president preference in each economic
range}\label{the-ratio-of-president-preference-in-each-economic-range}}

First, we compare voters' voting tendency in each financial section
toward president candidates \textbf{Donald Trump} and \textbf{Joe
Biden}.

As shown in the following figure, the red bar represented Donald Trump,
the blue bar represented Joe Biden, and the yellow bar and gold bar are
someone else and not vote. Moreover, we change the y axis to ratio to
clarify the percentage of the voting preference.

We can observe a phenomenon that if the people's annual income is lower
than \$10,000, they tend to untrust all candidates and try to seek help
for others. Therefore, there is an opportunity to both candidates that
\textbf{they can quickly figure out ways to win votes from those who
have no political preferences}.

On the other hand, people who have higher income prefer Donald Trump.
This situation may be due to factors related to tax cuts in 2017 (we
will figure it out later). But Donald Trump needs to figure out what
policies should be proposed to benefit the general public to win more
votes. In Joe Biden's viewpoint, he needs to figure out how to make rich
people more inclined to support him.

\includegraphics{Proj_1_files/figure-latex/vote analysis-1.pdf}

\begin{center}\rule{0.5\linewidth}{0.5pt}\end{center}

\hypertarget{the-ratio-of-party-preference-in-each-economic-range}{%
\subsection{The ratio of party preference in each economic
range}\label{the-ratio-of-party-preference-in-each-economic-range}}

Second, we compare voters' voting tendency in each financial section
toward political parties \textbf{Democrat} and \textbf{Republican}.

As shown in the following figure, we can further confirm our idea about
the relation between people with different income and their political
preference. The red bar represented Republican, the blue bar represented
Democrat, and the yellow bar, gold bar, and white bar are others else,
not vote, and don't know.

\includegraphics{Proj_1_files/figure-latex/unnamed-chunk-6-1.pdf}

\begin{center}\rule{0.5\linewidth}{0.5pt}\end{center}

\hypertarget{question-3-how-to-use-economic-policies-to-change-peoples-preference}{%
\subsection{Question 3: How to use economic policies to change people's
preference?}\label{question-3-how-to-use-economic-policies-to-change-peoples-preference}}

According to the analysis above, we can think about earning people's
hearts according to their financial situation and their preference using
politics!

\hypertarget{does-2017-law-that-reduced-federal-tax-rates-for-individuals-and-businesses-helped-or-hurt-the-nations-economy}{%
\subsubsection{Does 2017 law that reduced federal tax rates for
individuals and businesses helped or hurt the nation's
economy?}\label{does-2017-law-that-reduced-federal-tax-rates-for-individuals-and-businesses-helped-or-hurt-the-nations-economy}}

We can see that if the tax rate is lowered, people will be happier.
Therefore, if the candidate wants to earn people's votes, he needs to
take care of the tax rate policies.

\includegraphics{Proj_1_files/figure-latex/eco-1.pdf}

Next, we delve into the preference for the tax rate in people with
different income range. We can find that wealthier people intend to
favour the policy about reducing the tax rate. Combining with the Q2, we
can see that lowering the taxing rate will make rich people prefer to
vote, and that is why Donald Trump has a higher voting rate in the
high-income range.

On the other hand, people who have lower annual income tend to oppose
this policy. Therefore, if the candidate wants to earn the vote from
these people, lowering the tax rate might not be the right choice.

\includegraphics{Proj_1_files/figure-latex/unnamed-chunk-7-1.pdf}

\begin{center}\rule{0.5\linewidth}{0.5pt}\end{center}

\hypertarget{rich-people-need-to-pay-more-tax}{%
\subsubsection{Rich people need to pay more
tax?}\label{rich-people-need-to-pay-more-tax}}

Because people with lower incomes don't like the policy about lowering
the tax rate, we need to figure out another way to earn their votes!

Therefore, according to the figure below, we find that people favour
taxing the rich people who make \$10 million a year or make money more
than at a 70\% rate people.

\includegraphics{Proj_1_files/figure-latex/tax-1.pdf}

Next, we also delve into the policy's preference related to tax more on
rich people in different income ranges. Finally, we found that the
people in almost all income range tend to support this policy. We can
see that although some people who favour this policy are in the higher
income range, the field are between \$9,5000 - \$125,000, lower than the
target of the policy who need to have the annual income higher than
\textbf{\(10,000,000\)}.

\includegraphics{Proj_1_files/figure-latex/unnamed-chunk-9-1.pdf}

\hypertarget{finally-we-can-announce-that-the-candidate-who-want-to-earn-more-votes-taxing-more-on-the-people-who-make-10-million-a-year}{%
\paragraph{Finally, we can announce that the candidate who want to earn
more votes, taxing more on the people who make \$10 million a
year!}\label{finally-we-can-announce-that-the-candidate-who-want-to-earn-more-votes-taxing-more-on-the-people-who-make-10-million-a-year}}

\end{document}
